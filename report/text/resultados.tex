
\section{Projeto do motor}

Da metodologia e dos requisitos expostos na seção~\ref{sec:motor_project}, foi gerada a geometria do motor para impressão 3D. A figura~\ref{fig:internal_profile} mostra a seção longitudinal projetada para o motor. Destaca-se o grande volume da câmara em comparação com a tubeira, devido à necessidade de haver estagnação de um fluxo intenso (ver seção~\ref{sec:result_validation}).

\begin{figure}[htbp]
    \centering
    \includegraphics[width=\textwidth]{img/internal_profile.png}
    \caption{Perfil interno do motor projetado.}\label{fig:internal_profile}
\end{figure}

Os parâmetros propulsivos calculados para o motor foram:
\begin{itemize}
    \item \( \varepsilon = 1,35 \)
    \item \(C^* = 427,2\;\mathrm{m}\,\mathrm{s}^{-1}\)
    \item \(C_{F} = 1,10\)
\end{itemize}

Observa-se que o pequeno valor de razão de expansão é refletido na pequena tubeira exibida na figura~\ref{fig:internal_profile}. Foi obtido também um impulso específico de \(I_{sp} = 47,9s\) e um fluxo mássico de \(\dot{m} = 4,257\;\mathrm{g}\,\mathrm{s}^{-1}\). Devido à baixa energia do propelente, a velocidade característica do motor é bastante baixa, de modo que é necessário um fluxo mássico bastante elevado para a produção do empuxo requisitado. Isso é refletido no valor de impulso específico baixo. Ressalta-se também que o motor foi ajustado para operar à pressão ambiente, fator que limitou a razão de expansão e, portanto, o coeficiente de empuxo. 

O perfil da figura~\ref{fig:internal_profile} foi revolucionado para se obter uma geometria tridimensional, com superfícies planas externas adicionadas para facilitar a manufatura e a conexão com mangueiras de gás. Na figura~\ref{fig:3d_geom}, à esquerda, observa-se a geometria STL gerada em código para o motor. Foi adicionado um furo lateral para conexão com uma mangueira de alimentação, bem como quatro superfícies planas laterais para facilitar o manuseio e o apoio em morsas e outros equipamentos. À direita, observa-se o motor real construído. Destacam-se aqui as ranhuras da rosca para conexão da mangueira de gás.

\begin{figure}[htbp]
    \centering
    \begin{subfigure}{0.49\textwidth}
        \includegraphics[width=\textwidth]{img/motor_stl.png}
        \caption{Geometria STL gerada para o motor.}
    \end{subfigure}
    \begin{subfigure}{0.49\textwidth}
        \includegraphics[width=\textwidth]{img/motor_real.png}
        \caption{Motor impresso em 3D.}
    \end{subfigure}
    \caption{Geometria tridimensional do motor projetado.}\label{fig:3d_geom}
\end{figure}

\section{Validação do projeto do motor}\label{sec:result_validation}

A bancada de testes descrita na seção~\ref{sec:method_validation} gerou dados de empuxo para o motor e pressão estática e temperatura de câmara. A temperatura manteve-se bastante constante, próxima do valor especificado no requisito PRP-3, de modo que ela será tratada como constante e idêntica a este valor. O gráfico~\ref{fig:thrust_versus_p1} exibe as medidas de empuxo feitas em função da pressão de câmara medida.

\begin{figure}[htbp]
    \centering
    \includegraphics[width=\textwidth]{img/thrust_vs_chamber_pressure.png}
    \caption{Empuxo em função de pressão de câmara estática.}\label{fig:thrust_versus_p1}
\end{figure}

A variação de pressão de câmara observada neste ensaio deve-se ao grande fluxo mássico exigido pelo motor, de modo que houve esvaziamento significativo do compressor de ar utilizado durante o teste. Também destaca-se que a pressão medida na câmara, entre \(4,2bar\) e \(4,6bar\), foi inferior à pressão regulada em válvula, de exatamente \(5bar\). No entanto, o fato do motor real atingir os \(2N\) de empuxo exigidos com pressão próxima de \(4,5bar\) permitiu que a pesquisa continuasse levando-se em consideração as diferenças entre o motor real e o motor projetado.

Parâmetros propulsivos reais podem ser obtidos dos dados apresentados na figura~\ref{fig:thrust_versus_p1} através das equações~\ref{eq:C_F} e~\ref{eq:cstar}. Seus valores médios, com incerteza \(1\sigma \) são apresentados a seguir. A \(2N\) de empuxo, obteve-se impulso específico de \(46,6s\).
\begin{align}
    C_F &= 1,228 \pm 0,005 \\
    C^* &= (368,8 \pm 2,4)\;\mathrm{m}\,\mathrm{s}^{-1}
\end{align}

\section{Projeto do sistema de \textit{jet vanes}}\label{sec:result_jet_vanes}

Foi projetada uma montagem de suporte para o motor foguete, para o servomotor e para o mecanismo da lâmina defletora. A figura~\ref{fig:jet_vanes_render1} mostra este mecanismo em posição neutra. À esquerda tem-se o servomotor, acoplado a um braço extensor, que se conecta mecanicamente com uma peça de suporte da lâmina defletora. A lâmina é suportada em ambas as extremidades para garantir rigidez estrutural à peça. O suporte da lâmina defletora pode girar ao redor de seu centro, com um mancal visível no canto superior direito da figura. Logo abaixo deste suporte, é possível observar o furo de conexão com o eixo da balança de três componentes, descrita anteriormente. A peça alta imediatamente à direita do servomotor é o suporte para o motor foguete, cuja tubeira fica próxima à lâmina defletora. O recesso semicircular no topo desta peça permite a conexão do motor com uma mangueira de alimentação de gás frio.

\begin{figure}[htbp]
    \centering
    \includegraphics[width=\textwidth]{img/tvc_assembly_render1.png}
    \caption{Montagem do sistema de \textit{jet vanes} renderizada no CATIA.}\label{fig:jet_vanes_render1}
\end{figure}

A figura~\ref{fig:jet_vanes_render2} demonstra o funcionamento do mecanismo projetado em CAD\@. Foram impostos à montagem em CAD \textit{constraints} mecânicos que simulam o comportamento pretendido do sistema, de modo que os graus de liberdade das peças estejam adequadamente representados. O ângulo de rotação do servomotor é transmitido ao suporte da lâmina defletora através de uma haste longitudinal. Por projeto, os raios dos encaixes desta haste no servomotor e no suporte da lâmina defletora são iguais, de modo que os ângulos de deflexão são idênticos. O extensor do servomotor foi calibrado para que apresentasse posição perpendicular ao eixo longitudinal da montagem quando o servomotor recebesse um comando de \(90\mathrm{^\circ}\). O comprimento da haste foi estabelecido de modo que o suporte da lâmina defletora ficasse paralelo ao servomotor.

\begin{figure}[htbp]
    \centering
    \includegraphics[width=\textwidth]{img/tvc_assembly_render2.png}
    \caption{Sistema de \textit{jet vanes} defletido de \(20\mathrm{^\circ}\).}\label{fig:jet_vanes_render2}
\end{figure}

As peças projetadas foram então impressas em 3D e montadas. A figura~\ref{fig:jet_vanes_assembly_side} mostra uma vista lateral do mecanismo. Destaca-se aqui o motor foguete montado em seu suporte, com um conector de mangueira de gás frio acoplado. Destaca-se também o eixo de metal à direita, usado para o encaixe na balança de três componentes utilizada.

\begin{figure}[htbp]
    \centering
    \includegraphics[width=\textwidth]{img/tvc_assembly_left.jpeg}
    \caption{Vista lateral do sistema de \textit{jet vanes} impresso e montado.}\label{fig:jet_vanes_assembly_side}
\end{figure}


\section{Caracterização do sistema em balança de três componentes}\label{sec:result_characterization}

As curvas de calibração dos componentes da balança estão exibidas na figura~\ref{fig:calibration_curves}. A curva vermelha tracejada indica a função identidade, ou seja, a igualdade exata entre a força calibrada e a força aplicada durante a calibração. Os pontos azuis indicam a força calculada através da calibração feita, usando a saída da balança para um dado carregamento. Nesta seção, todas as barras de erro referem-se ao erro quadrático médio obtido ao longo de \(1s\) de medida com amostragem de \(1000\mathrm{Hz}\). A calibração foi feita com dados de carregamento e descarregamento da balança para que a histerese dos componentes fosse observável, como na figura~\ref{fig:calibration_FF}. Destaca-se também que devido ao posicionamento do sistema, a força horizontal esperada era simétrica ao redor de zero, de modo que a componente \(F_D\) foi calibrada simetricamente ao redor de zero. Observa-se também que o zero das curvas apresentadas é referenciado em relação ao pré-carregamento aplicado, conforme explicado na seção~\ref{sec:method_3axis_measurement}.

\begin{figure}[htbp]
    \centering
    \begin{subfigure}{\textwidth}\centering
        \includegraphics[height=.29\textheight]{img/results/calibration_FA.png}
        \caption{Componente \textit{aft}.}\label{fig:calibration_FA}
    \end{subfigure}
    \begin{subfigure}{\textwidth}\centering
        \includegraphics[height=.29\textheight]{img/results/calibration_FF.png}
        \caption{Componente \textit{fore}.}\label{fig:calibration_FF}
    \end{subfigure}
    \begin{subfigure}{\textwidth}\centering
        \includegraphics[height=.29\textheight]{img/results/calibration_FD.png}
        \caption{Componente \textit{drag}.}\label{fig:calibration_FD}
    \end{subfigure}
    \caption{Curvas de calibração dos componentes da balança.}\label{fig:calibration_curves}
\end{figure}

Os resultados de empuxo sem lâmina defletora estão exibidos na figura~\ref{fig:thrust_no_deflector}. A primeira linha vermelha tracejada, à esquerda, indica o acionamento do compressor de ar. A segunda, indica seu desligamento, e por fim a terceira seu acionamento final. O compressor de ar usado aciona-se automaticamente quando a pressão do reservatório (400L) reduz-se abaixo de certo limite (8bar). Assim, esse experimento demonstra que as variações de pressão no reservatório pouco afetam o experimento, conectado ao reservatório de ar por uma válvula reguladora. A queda de empuxo ao final do experimento ocorreu após quase um minuto de coleta de dados, de modo que foi comprovada a capacidade de fornecimento de ar do compressor. Com estes dados, e também o valor do coeficiente de empuxo empírico, conclui-se que a pressão real de câmara obtida foi de \(5,60\pm 0,15\;\mathrm{bar}\). A diferença deste valor para os valores obtidos no gráfico~\ref{fig:thrust_versus_p1} deve-se à mudança do sistema de compressão de ar utilizado.

\begin{figure}[htbp]
    \centering
    \includegraphics[width=\textwidth]{img/results/thrust_no_deflector.png}
    \caption{Medidas de empuxo sem lâmina defletora.}\label{fig:thrust_no_deflector}
\end{figure}

Finalmente, apresentam-se os gráficos de forças e momento em função da deflexão do servomotor. Recorda-se aqui que a posição de \(90\mathrm{^{\circ}}\) do servomotor corresponde ao paralelismo da lâmina defletora com o escoamento da tubeira. Como mencionado na seção~\ref{sec:method_3axis_measurement}, foram feitas três repetições de varreduras crescentes e decrescentes de posição. Os gráficos de \(F_x\), \(F_y\) e \(M\) obtidos após conversão das saídas dos componentes para forças aplicadas (através da calibração e equações~\ref{eq:FA} a~\ref{eq:FD}) estão exibidos na figura~\ref{fig:deflection_forces}.

\begin{figure}[htbp]
    \centering
    \begin{subfigure}{0.49\textwidth}
        \includegraphics[width=\textwidth]{img/results/exp7_5bar_70_a_110.png}
    \end{subfigure}
    \begin{subfigure}{0.49\textwidth}
        \includegraphics[width=\textwidth]{img/results/exp8_5bar_110_a_70.png}
    \end{subfigure}
    \begin{subfigure}{0.49\textwidth}
        \includegraphics[width=\textwidth]{img/results/exp9_5bar_70_a_110.png}
    \end{subfigure}
    \begin{subfigure}{0.49\textwidth}
        \includegraphics[width=\textwidth]{img/results/exp10_5bar_110_a_70.png}
    \end{subfigure}
    \begin{subfigure}{0.49\textwidth}
        \includegraphics[width=\textwidth]{img/results/exp11_5bar_70_a_110.png}
    \end{subfigure}
    \begin{subfigure}{0.49\textwidth}
        \includegraphics[width=\textwidth]{img/results/exp12_5bar_110_a_70.png}
    \end{subfigure}
    \caption{Curvas de força longitudinal, força transversal e momento medidas para as varreduras de deflexão.}\label{fig:deflection_forces}
\end{figure}

Os gráficos à esquerda correspondem às varreduras no sentido \(70\mathrm{^{\circ}} \rightarrow 110\mathrm{^\circ}\) e os à direita, às varreduras no sentido contrário. Se houvesse folga mecânica nos componentes, seria esperado haver discordância entre os gráficos da esquerda e da direita, devido à mudança de sentido da força sofrida pelo sistema defletor ao passar pela posição \(90\mathrm{^{\circ}}\). Havendo, pelo contrário, forte semelhança entre os gráficos, é possível excluir imperfeições mecânicas como fonte de erros para o sistema. Observa-se aqui que o momento apresentado nos gráficos refere-se ao momento ao redor do eixo da balança (localizado nas figuras~\ref{fig:jet_vanes_assembly_side},~\ref{fig:jet_vanes_render1} e~\ref{fig:montagem_interna}).

As curvas de força transversal e momento apresentam comportamento linear para pequenas deflexões. Observa-se, em todos os gráficos, uma mudança no coeficiente angular destas quantidades por volta dos \(115\mathrm{^{\circ}}\), o que corresponde ao término da região linear prevista pela equação~\ref{eq:sslift}. Nota-se no entanto que na posição \(90\mathrm{^\circ}\) há forças e momentos residuais, o que não condiz com a simetria esperada para a configuração. Esta assimetria pode ser atribuída a erros experimentais introduzidos pelo equipamento, como por exemplo a rigidez da mangueira de gás, alterada pelo escoamento a alta pressão em seu interior, como discutido na seção~\ref{sec:method_3axis_measurement}. Observa-se também que devido ao corte da lâmina defletora com tesouras de metal, seu perfil também é um pouco assimétrico. O fato destes valores serem semelhantes para todos os experimentos conduzidos corrobora a hipótese de que são fruto de erros experimentais sistemáticos e podem, portanto, ser desconsiderados.

Quanto à curva de força longitudinal ou empuxo, \(F_y\), percebe-se que houve uma redução consistente de empuxo em relação ao empuxo sem lâmina defletora da figura~\ref{fig:thrust_no_deflector}. A introdução de uma lâmina de espessura não nula gera ondas de choque no escoamento supersônico mesmo a ângulo de ataque nulo~\cite{anderson}, que reduzem a velocidade do escoamento e portanto o empuxo do motor. Observa-se também um pico de empuxo próximo à deflexão de \(100\mathrm{^\circ}\), correlacionado com o momento durante o experimento no qual o compressor foi reativado devido ao seu esvaziamento. Assim, esta curva apresenta erros experimentais significativos, de modo que quaisquer variações reais de empuxo em função da deflexão da lâmina são indetermináveis.

Tendo sido exibidos os gráficos diretamente obtidos dos dados experimentais, e tendo sido constatada sua semelhança, extrair-se-ão parâmetros médios para o sistema. O empuxo médio obtido foi 
\begin{equation}
    F_{av} = 1,77 \pm 0,02\;\mathrm{N}
\end{equation}
correspondendo a uma redução de 41\% em relação ao empuxo sem lâmina defletora, o que leva a considerações de projeto importantes em relação à eficiência. É possível também calcular derivadas de controle médias para a força transversal e para o momento:
\begin{align}
    F_{x\delta} &= (3,57 \pm 0,05)\times10^{-2} \mathrm{N} / \mathrm{^\circ} \\
    M_\delta &= (-2,29 \pm 0,02)\times10^{-2} \mathrm{N}\,\mathrm{cm} / \mathrm{^\circ}
\end{align}
onde \(\delta \) representa a deflexão da lâmina. O momento é dado em relação ao eixo da balança.\ 

\section{Discussão sobre a metodologia}

O estudo sobre vetorização de empuxo é uma área promissora mas nova no Brasil. Desta forma, este trabalho foi uma primeira aproximação do assunto, no qual foram feitas suposições sobre a metodologia adequada para estudar o tema. Agora, avaliar-se-ão algumas das escolhas feitas, considerando a possibilidade de trabalhos futuros.

A escolha do gás frio como propelente deve-se à sua temperatura, que permite o uso de componentes impressos em 3D, bem como à sua disponibilidade e custo. Outra característica do gás frio constatada no trabalho é sua reprodutibilidade: o empuxo é estável (figura~\ref{fig:thrust_no_deflector}) e não varia entre ativações do motor (figura~\ref{fig:deflection_forces}). Todas essas são características desejáveis e provaram-se necessárias para a execução do trabalho. Supondo, por exemplo, que o trabalho tivesse sido conduzido com um motor de combustível sólido, introduzir-se-iam variações devido à mistura dos propelentes e devido à curva de empuxo do motor~\cite{Sutton}, além de introduzir-se a necessidade de manufaturar todos os componentes em materiais resistentes à temperatura. Estima-se que o motor tenha sido ativado por cerca de \(20\mathrm{min}\) ao longo do trabalho, de modo que seria necessário fabricar vários grãos-propelente para a execução dos testes necessários.

No entanto, a baixa energia do propelente significa que uma alta vazão mássica é necessária mesmo para os pequenos empuxos usados no trabalho. Originalmente almejava-se usar um motor de \(5\mathrm{N}\), mas este provou-se inviável devido à velocidade de consumo de ar comprimido. Ao longo do projeto foram utilizados cilindros de nitrogênio comprimido e dois compressores de ar distintos (40L e 8bar e 400L e 9bar). O cilindro de nitrogênio, de \(200\mathrm{bar}\) e 50L foi esvaziado rapidamente, e o primeiro compressor apresentava uma rápida queda de pressão (motivo pelo qual o gráfico~\ref{fig:thrust_no_deflector} apresenta pressão variável). O maior compressor durou mais tempo, mas possivelmente introduziu erros nos gráficos da~\ref{fig:deflection_forces}. Dessa forma, para trabalhos futuros, garantir a compatibilidade do sistema propulsivo com o sistema de fornecimento de ar comprimido é fundamental.\ 

Quanto à balança de três componentes usada, faz-se necessário encontrar alternativas para trabalhos futuros. A balança é montada a um túnel de vento, o que restringe a montagem do sistema propulsivo. Isto impediu, por exemplo, a montagem de um transdutor de pressão na câmara de empuxo nas medidas finais, já que essa montagem introduziria ainda mais erros devidos à rigidez da mangueira. Tendo sido adquirida para medir forças da escala de 5 a 20N~\cite{lab}, ela mostra-se inadequada para as forças permitidas pelos sistemas de fornecimento de ar comprimido disponíveis. Este aparato não é de simples fabricação, sendo necessária precisão micrométrica para alguns componentes. No entanto, parece ser um passo necessário para o desenvolvimento de sistemas de vetorização de empuxo no Brasil a fabricação de uma bancada de teste de empuxo com balança de três componentes (para sistemas com um grau de liberdade, que geram forças e momento em um plano) ou seis componentes (para medição completa de forças e momentos tridimensionais), específicas para o propósito.\

Por fim, 