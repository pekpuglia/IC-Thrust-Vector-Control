Agradeço ao professor Leonardo Gouvêa por, certo dia falando sobre qualquer coisa, levantar a possibilidade deste trabalho, e por auxiliar no que é necessário auxílio e dar liberdade no que é necessário liberdade.

Agradeço também ao João Vitor Baldo e Arthur Zoppi, que acompanharam todas as minhas impressões 3D no Laboratório Aberto do CCM-ITA, sempre dando dicas e sugestões que iam além do esperado deles. Obrigado por transformar impressões problemáticas e fracassadas em momentos de descontração e aprendizado sobre mundo real.

Agradeço por fim a toda a equipe do Laboratório Feng, que foi recrutada por acaso no meio do caminho para minha iniciação científica. Agradeço ao Prof. Dr. Tiago Barbosa, que permitiu o uso do laboratório, e ao professor André Fernando de Castro, que certo dia por acaso resolveu (quase) todos os problemas experimentais do meu trabalho comigo. Agradeço especialmente aos técnicos Newton, que me auxiliou com toda a montagem mecânica do experimento, e Wilson, que me acompanhou pelas várias horas de montagem e calibração.

Por fim agradeço à minha família, que sempre me incentivou, e que inúmeras vezes me ouviu falar sobre os detalhes mais desinteressantes deste trabalho.