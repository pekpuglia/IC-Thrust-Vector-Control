Este trabalho apresenta o processo de desenvolvimento e caracterização de um sistema de vetorização de empuxo com motor a gás frio. O motor tem como requisito empuxo de \(2\;\mathrm{N}\) e \(5\;\mathrm{bar}\) de pressão de câmara. O método de vetorização escolhido para teste foi o de \textit{jet vane}. O motor construído apresentou divergências pequenas com os requisitos, tendo um impulso específico de \(46,6\;\mathrm{s}\). Este motor foi montado em um mecanismo de controle da lâmina defletora e esta montagem foi acoplada a uma balança de três componentes para caracterização das forças e momentos gerados. Como resultado final, obtiveram-se as derivadas de controle de força lateral e momento. Por fim, apresentaram-se os problemas metodológicos encontrados e \textit{trade-offs} de engenharia identificados para o sistema.