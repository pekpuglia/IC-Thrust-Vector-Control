%%% Exemplo de utilização da classe ITA
%%%
%%%   por        Fábio Fagundes Silveira   -  ffs [at] ita [dot] br
%%%              Benedito C. O. Maciel     -  bcmaciel [at] ita [dot] br
%%%              Giovani Volnei Meinertz   -  giovani [at] ita [dot] br
%%%    	         Hudson Alberto Bode       -  bode [at] ita [dot]br
%%%    	         P. I. Braga de Queiroz    -  pi [at] ita [dot] br
%%%    	         Jorge A. B. Gripp         -  gripp [at] ita [dot] br
%%%    	         Juliano Monte-Mor         -  jamontemor [at] yahoo [dot] com [dot] br
%%%    	         Tarcisio A. B. Gripp      -  tarcisio.gripp [at] gmail [dot] com
%%%
%%%   Versão para overleaf:
%%%   por        Alejandro A. Rios Cruz    - aarc.88@gmail.com
%%%              Saulo Gómez               - sagomezs@unal.edu.co
%%%              Ocimar Santos             - ocimar.acad@gmail.com
%%%
%%%   Template disponibilizado em:
%%%              Overleaf: https://pt.overleaf.com/latex/templates/thesis-template-aeronautics-institute-of-technology-ita/yhfrqqydpygk
%%%
%%%   Contribuia você também!
%%%              GitHub:   https://github.com/AlejandroRios/Template_Thesis_ITA
%%%
%%%  IMPORTANTE: O texto contido neste exemplo nao significa absolutamente nada.  :-)
%%%              O intuito aqui eh demonstrar os comandos criados na classe e suas
%%%              respectivas utilizacoes.
%%%
%%%  Tese.tex  2016-08-25
%%%  $HeadURL: http://www.apgita.org.br/apgita/teses-e-latex.php $
%%%
%%% ITALUS
%%% Instituto Tecnológico de Aeronáutica --- ITA, Sao Jose dos Campos, Brasil
%%%                   http://groups.yahoo.com/group/italus/
%%% Discussion list: italus {at} yahoogroups.com
%%%
%++++++++++++++++++++++++++++++++++++++++++++++++++++++++++++++++++++++++++++++
% Para alterar o TIPO DE DOCUMENTO, preencher a linha abaixo \documentclass[?]{?}
%   \documentclass[tg]{ita}			= Trabalho de Graduacao
%   \documentclass[tgfem]{ita}	= Para Engenheiras
%   								msc     		= Dissertacao de Mestrado
%   								mscfem   		= Para Mestras
%   								dsc      		= Tese de Doutorado
%   								dscfem   		= Para Doutoras
%   								quali    		= Exame de Qualificacao
%   								qualifem 		= Exame de Qualificacao para Doutoras
% Para 'Draft Version'/'Versao Preliminar' com data no rodape, adicionar 'dv':
%   \documentclass[dsc, dv]{ita}
% Para trabalhos em Inglês, adicionar 'eng':
%   \documentclass[dsc, eng]{ita}
%		\documentclass[dsc, eng, dv]{ita}
%++++++++++++++++++++++++++++++++++++++++++++++++++++++++++++++++++++++++++++++
\documentclass[tg]{ita}    % ITA.cls based on standard book.cls
% Quando alterar a classe, por exemplo de [msc] para [msc, eng]) rode mais uma vez o botão BUILD OUTPUT caso haja erro
\usepackage[utf8]{inputenc}
\usepackage{ae}
\usepackage{graphicx}
\usepackage{epsfig}
\usepackage{amsmath}
\usepackage{amssymb}
\usepackage{subcaption}
\usepackage{multirow}
\usepackage{float}
\usepackage{amsthm}
\usepackage{url}         % formats URL addresses properly
\usepackage{appendix}    % allows appendix section to be included
\usepackage{lscape}      % allows a page to be rendered in landscape mode
\usepackage{multicol}    % allows text in multi columns
\usepackage{cancel}      % needed to show canceled terms in equations
\usepackage{lettrine}
\usepackage{float}
\usepackage{placeins}


%HHHHHHHHHHHHHHHHHHHHHHHHHHHHHHHHHHHHHHHHHHHHHHHHHHHHHHHHHHHHHHHHHHHHHHHHHHHHHHHHHHHHHHHHHHHHHHHHHHHHHHHHHHHH
%\usepackage{subfigure}
%\usepackage{subfigmat}
%PACOTEFIGURAS_SE _ERRADO_ESXCLUIR_ACIMA
\usepackage{booktabs}
%PACOTETABELAS_SE _ERRADO_ESXCLUIR_ACIMA
%HHHHHHHHHHHHHHHHHHHHHHHHHHHHHHHHHHHHHHHHHHHHHHHHHHHHHHHHHHHHHHHHHHHHHHHHHHHHHHHHHHHHHHHHHHHHHHHHHHHHHHHHHHHH

%++++++++++++++++++++++++++++++++++++++++++++++++++++++++++++++++++++++++++++++
% Espaçamento padrão de todo o documento
%++++++++++++++++++++++++++++++++++++++++++++++++++++++++++++++++++++++++++++++
\onehalfspacing

%singlespacing Para um espaçamento simples
%onehalfspacing Para um espaçamento de 1,5
%doublespacing Para um espaçamento duplo

%++++++++++++++++++++++++++++++++++++++++++++++++++++++++++++++++++++++++++++++
% Identificacoes (se o trabalho for em inglês, insira os dados em inglês)
% Para entradas abreviadas de Professora (Profa.) em português escreva: Prof$^\textnormal{a}$.
%++++++++++++++++++++++++++++++++++++++++++++++++++++++++++++++++++++++++++++++
\course{Engenheria Aeroespacial}  % Programa de PG ou Curso de Graduação
%\area{Aircraft Design} % Área de concentração na PG (Não utilizado no caso de TG)

% Autor do trabalho: Nome Sobrenome
\authorgender{masc}                     %sexo: masc ou fem
\author{Pedro Kuntz}{Puglia}
\itaauthoraddress{Rua H8C, Ap. 303}{12.228- 462}{São José dos Campos- SP}

% Titulo da Tese/Dissertação
\title{Caracterização de sistema de propulsão a gás frio com vetorização de empuxo}

% Orientador
\advisorgender{masc}                    % masc ou fem
\advisor{Prof.~Dr.}{Leonardo Gouvêa}{ITA}

% Coorientador (Caso não haja coorientador, colocar ambas as variáveis \coadvisorgender e \coadvisor comentadas, com um % na frente)
\coadvisorgender{masc}									% masc ou fem
\coadvisor{Prof.~Dr.}{Maurício Morales}{ITA}

% Pró-reitor da Pós-graduação
% \bossgender{masc}												% masc ou fem
% \boss{Prof.~Dr.}{John von Neumann}

%Coordenador do curso no caso de TG
\bosscoursegender{fem}									% masc ou fem
\bosscourse{Profa.~Dra.}{Cristiane Martins}

% Palavras-Chaves informadas pela Biblioteca -> utilizada na CIP
\kwcip{Propulsão}
\kwcip{Gás Frio}
\kwcip{Vetorização de empuxo}

% membros da banca examinadora

% \examiner{Prof. Dr.}{Alan Turing}{Presidente}{ITA}
% \examiner{Prof. Dr.}{Linus Torwald}{}{UXXX}
% \examiner{Prof. Dr.}{Richard Stallman}{}{UYYY}
% \examiner{Prof. Dr.}{Donald Duck}{}{DYSNEY}
% \examiner{Prof. Dr.}{Mickey Mouse}{}{DISNEY}

% Data da defesa (mês em maiúsculo, se trabalho em inglês, e minúsculo se trabalho em português)
\date{11}{julho}{2023}

% Número CDU - (somente para TG)
\cdu{???.??}

% Glossario
\makeglossary
\frontmatter

\begin{document}
% Folha de Rosto e Capa para o caso do TG
\maketitle

% Dedicatoria: Nao esqueca essa secao  ... :-)
\begin{itadedication}
A todos que algum dia contribuíram ou contribuirão à ciência brasileira
\end{itadedication}

% Agradecimentos
\begin{itathanks}
Agradeço ao professor Leonardo Gouvêa por, certo dia falando sobre qualquer coisa, levantar a possibilidade deste trabalho, e por auxiliar no que é necessário auxílio e dar liberdade no que é necessário liberdade.

Agradeço também ao João Vitor Baldo e Arthur Zoppi, que acompanharam todas as minhas impressões 3D no Laboratório Aberto do CCM-ITA, sempre dando dicas e sugestões que iam além do esperado deles. Obrigado por transformar impressões problemáticas e fracassadas em momentos de descontração e aprendizado sobre mundo real.

Agradeço por fim a toda a equipe do Laboratório Feng, que foi recrutada por acaso no meio do caminho para minha iniciação científica. Agradeço ao Prof. Dr. Tiago Barbosa, que permitiu o uso do laboratório, e ao professor André Fernando de Castro, que certo dia por acaso resolveu (quase) todos os problemas experimentais do meu trabalho comigo. Agradeço especialmente aos técnicos Newton, que me auxiliou com toda a montagem mecânica do experimento, e Wilson, que me acompanhou pelas várias horas de montagem e calibração.

Por fim agradeço à minha família, que sempre me incentivou, e aos meus amigos de ITA e H8, que inúmeras vezes me ouviram falar detalhadamente sobre os problemas deste trabalho.
\end{itathanks}

% Epígrafe
\thispagestyle{empty}
\ifhyperref\pdfbookmark[0]{\nameepigraphe}{epigrafe}\fi
\begin{flushright}
\begin{spacing}{1}
\mbox{}\vfill
{\sffamily\itshape
``Pointy end up,
flamey end down.''\\}
--- \textsc{Tim Dodd, Everyday Astronaut}

\end{spacing}
\end{flushright}

% Resumo
\begin{abstract}
\noindent
Este trabalho apresenta o processo de desenvolvimento e caracterização de um sistema de vetorização de empuxo com motor a gás frio. O motor tem como requisito empuxo de \(2\;\mathrm{N}\) e \(5\;\mathrm{bar}\) de pressão de câmara. O método de vetorização escolhido para teste foi o de \textit{jet vane}. O motor construído apresentou divergências pequenas com os requisitos, tendo um impulso específico de \(46,6\;\mathrm{s}\). Este motor foi montado em um mecanismo de controle da lâmina defletora e esta montagem foi acoplada a uma balança de três componentes para caracterização das forças e momentos gerados. Como resultado final, obtiveram-se as derivadas de controle de força lateral e momento. Por fim, apresentaram-se os problemas metodológicos encontrados e \textit{trade-offs} de engenharia identificados para o sistema.
\end{abstract}

% Abstract
\begin{englishabstract}
\noindent
This work presents the development and characterization process of a cold gas thruster vectorization system. The motor is required to have a thrust of \(2\; \mathrm{N}\) and a chamber pressure of \(5\; \mathrm{bar}\). The chosen vectorization method for testing was the jet vane. The constructed motor had slight deviations from the requirements, with a specific impulse of \(46.6\; \mathrm{s}\). This motor was mounted on a control mechanism of the deflecting blade, and this assembly was coupled to a three-component scale for force and moment characterization. As a final result, the control derivatives for lateral force and moment were obtained. Finally, the methodological issues encountered and engineering trade-offs identified for the system were presented.
\end{englishabstract}

% Lista de figuras
\listoffigures %opcional

% Lista de tabelas
\listoftables %opcional

% Lista de abreviaturas
% \listofabbreviations
% \input{PreTextuais/listaabreviaturas} %opcional

% Lista de simbolos
\listofsymbols
\begin{longtable}{ll}
\(F\) & Empuxo propulsivo \\
\(\dot{m}\) & Vazão mássica \\
\(v_e\) & Velocidade de exaustão média \\
\(p_c\) & Pressão de câmara \\
\(p_e\) & Pressão de saída média \\
\(p_{amb}\) & Pressão ambiente \\
\(A_c\) & Área da seção transversal da câmara \\
\(A_e\) & Área da seção transversal da saída da tubeira \\
\(A_t\) & Área da seção transversal da garganta \\
\(\varepsilon \) & Razão de expansão \\
\(I_{\mathrm{sp}}\) & Impulso específico \\
\(C_F\) & Coeficiente de empuxo \\
\(C^*\) & Velocidade característica \\
\(F_x\) & Força horizontal, transversal ao motor foguete \\
\(F_y\) & Força vertical, na direção do empuxo propulsivo \\
\(M\) & Torque resultante \\
\(\delta \) & Deflexão da lâmina (\textit{jet vane}) \\
\(F_{x\delta}\) & Derivada da força lateral em relação à deflexão da lâmina \\
\(M_\delta \) & Derivada de momento em relação à deflexão da lâmina
\end{longtable}

 %opcional

% Sumario
\tableofcontents


\mainmatter

%todo
%resumo, abstract

% Os capitulos comecam aqui
\chapter{\textbf{INTRODUÇÃO}}
\section{Contexto histórico e motivação}

%http://heroicrelics.org/info/v-2/v-2-cut-away.html
%https://v2rockethistory.com/?s=jet+vane

A tecnologia de empuxo vetorial (ou TVC, do inglês \textit{thrust vector control}) é chave para o setor aeroespacial, pois permite aproveitar o empuxo gerado pelo motor-foguete para aplicar um comando de atitude ao veículo. É uma tecnologia desenvolvida desde os primórdios da tecnologia de foguetes, com o míssil V2 sendo um marco notável no histórico do empuxo vetorial e dos foguetes. Este sistema, exibido na figura~\ref{fig:tvc_systems_jet_vanes}, utilizava lâminas de grafite (\textit{jet vanes}) inseridas na exaustão do motor principal para direcionar o escoamento de gases e produzir uma força lateral capaz de direcionar o míssil.

\begin{figure}
    \centering
    \begin{subfigure}{.49\textwidth}
        \centering
        \includegraphics[width=.9\textwidth]{img/v2jetvanes.jpg}
        \caption{Sistema de \textit{jet vanes} do míssil V2~\cite{V2jetvanes}.}\label{fig:tvc_systems_jet_vanes}
    \end{subfigure}
    \hfill
    \begin{subfigure}{.49\textwidth}
        \centering
        \includegraphics[width=.9\textwidth]{img/engine_gimbal.png}
        \caption{Sistema de \textit{gimbal} do motor RS-25 do foguete Artemis~\cite{RS25Gimbal}.}\label{fig:tvc_systems_gimbal}
    \end{subfigure}
    \caption{Exemplos de sistemas com empuxo vetorial.}
\end{figure}

Outros sistemas de empuxo vetorial foram desenvolvidos após a Segunda Guerra Mundial, tanto para aplicações militares como para lançadores de satélites, cada uma com seus \textit{trade-offs} de engenharia. Uma alternativa de alto desempenho e alta complexidade mecânica muito comum atualmente é a articulação esférica, ou \textit{gimbal}, da tubeira do motor. Um sistema \textit{gimbal}, do motor RS-25 desenvolvido para o ônibus espacial e reaproveitado para o programa Artemis, é exibido em ação na figura~\ref{fig:tvc_systems_gimbal}.

Os sistemas de empuxo vetorial são fundamentais para a estabilidade e para o seguimento de trajetória dos foguetes. Defeitos de manufatura podem introduzir desalinhamentos angulares e lineares de empuxo, que devem ser compensados pelo sistema de controle de empuxo vetorial. Também são fundamentais para o controle dos veículos em baixas velocidades, regime no qual aletas fornecem pouca ou nenhuma autoridade sobre o veículo, permitindo que se elimine a necessidade de trilhos de lançamento. Naturalmente, também funcionam no vácuo espacial. Este trabalho busca, portanto, iniciar uma linha de pesquisa brasileira sobre o assunto.

\section{Objetivos}

Este trabalho buscou desenvolver motor foguete a gás frio de pequena escala (2--5N), um sistema de empuxo vetorial baseado em \textit{jet vane} para direcionamento do empuxo em um plano, e a caracterização empírica das forças geradas pelo sistema, bem como as dificuldades identificadas para o desenvolvimento futuro do tema.

\section{Introdução Teórica}

A propulsão de motores-foguete é baseada na força de reação gerada pela aceleração de uma massa em um sentido oposto ao sentido desejado da força propulsiva. Tecnologicamente, este conceito é implementado com o uso de escoamentos de fluidos compressíveis, acelerados por diferenças de pressão presentes no sistema. Do ponto de vista da engenharia aeroespacial, faz-se necessário conhecer algumas métricas de eficiência que possam ser aplicadas em um projeto. Assim, a propulsão é uma área de aplicação de mecânica, termodinâmica e engenharia.

\chapter{\textbf{METODOLOGIA}}

O trabalho foi iniciado com o projeto do motor a gás frio. A seguir, este motor foi validado experimentalmente em uma bancada de teste de empuxo. Com o sistema propulsivo pronto, projetou-se o sistema de \textit{jet vanes}. Por fim, este sistema foi aferido em uma balança de três componentes para medição do empuxo e força lateral gerada pelo sistema. 

\section{Projeto do motor}

O projeto do motor foi feito de maneira programática e iterativa, assegurando fácil reprodução dos resultados obtidos e automação do fluxo de dados. A linguagem de programação \textit{Julia} foi utilizada para o projeto. Nesta seção, serão apresentados os dados referentes à última versão do motor. Para um histórico do desenvolvimento do motor, consultar o apêndice AAAAAAAAAAA.  

A tabela~\ref{tab:requirements} mostra os requisitos propulsivos, codificados PRP-N, e geométricos, codificados GMT-N, levantados para o motor. Os requisitos PRP-1 e PRP-2 foram propostos com base nos sistemas de fornecimento de ar disponíveis e na escala desejada do motor. Já a temperatura do propelente, requisito PRP-2, é baseada na temperatura ambiente, e permite a utilização de \textit{jet vanes} feitas de materiais simples. Com base nestes requisitos, um sistema monopropelente a ar foi proposto. Os requisitos GMT-1 e GMT-2 foram especificados com base na necessidade de haver estagnação na câmara de empuxo (seção~\ref{sec:intro}) e na necessidade de fácil manipulação, manufatura e conexão. Já os requisitos GMT-3 e GMT-4 buscam propiciar um escoamento com alto paralelismo na região da tubeira, assim como facilitar a manufatura.

\begin{table}[]
    \centering\begin{tabular}{cccc} \toprule
        Código & Variável & Grandeza & Valor \\ \midrule
        PRP-1 & \(P_C\) & Pressão de câmara & \(500kPa\) \\
        PRP-2 & \(T\) & Empuxo & \(2N\) \\
        PRP-3 &\(T_{prop}\) & Temperatura do propelente & \(298,15K\) \\
        GMT-1 & \(r_{C,\text{min}}\) & Raio de câmara mínimo & \(15mm\) \\
        GMT-2 & \(L_C\) & Comprimento de câmara & \(30mm\) \\
        GMT-3 & \(\alpha_{\text{conv}}\) & Semi-ângulo do convergente & \(30^\circ \) \\
        GMT-4 & \(\alpha_{\text{div}}\) & Semi-ângulo do divergente & \(5^\circ \) \\ \bottomrule 
    \end{tabular}
    \caption{Requisitos propulsivos e geométricos para o motor.}
    \label{tab:requirements}
\end{table}

A partir destes requisitos, o software CEA NASA foi utilizado para calcular os parâmetros propulsivos (\(\varepsilon \), \(C^\ast \) e \(C_f\)) do sistema com pressão ambiente \(P_{amb} = 100kPa\). Com estes coeficientes, pode-se aplicar as relações~\ref{eq:exp_ratio} e~\ref{eq:C_F} descritas na seção~\ref{sec:intro} para calcular as áreas de saída \(A_e\), de garganta \(A_t\). A área de câmara, \(A_C\), foi calculada diretamente a partir do requisito GMT-1. Foi introduzida uma seção cilíndrica na garganta do motor, com área de seção transversal \(A_t\), para garantir a manufatura precisa dessa dimensão.

Os três coeficientes propulsivos calculados também foram utilizados para estimar o fluxo mássico de propelente \(\dot{m}\), e este, para estimar a velocidade do propelente na mangueira de alimentação, \(v_{\text{prop}}\). Estes parâmetros são relevantes para a verificação da perda de carga no sistema de alimentação, bem como para a escolha da fonte de ar. Como o propelente é pouco energético, altas vazões são necessárias mesmo para empuxos pequenos, de modo que conhecer a capacidade exigida da fonte foi fundamental. A partir de \(\dot{m}\), cujo cálculo foi descrito anteriormente, e assumindo que a  um diâmetro de tubo \(d\), massa molar do ar \(MM_{ar}\) e constante dos gases \(R\):
\begin{equation}
    v_{\text{prop}} = \frac{\dot{m} R T_{\text{prop}}}{\pi \left(\frac{d}{2}\right)^2 P_C MM_{ar}}
\end{equation}

Com as áreas das seções transversais do motor calculadas, e em posse da especificação da geometria interna do motor, gerou-se o CAD do motor para impressão 3D. Em \textit{Julia}, utilizou-se a package \textit{ConstructiveGeometry}\footnote[1]{https://github.com/plut/ConstructiveGeometry.jl} para gerar a geometria tridimensional a partir do dados de geometria calculados. Ao produzir o CAD diretamente a partir do código de projeto, foi possível eliminar etapas manuais que podem introduzir erros e atrasos ao projeto. A impressão 3D foi escolhida como método de manufatura devido à baixa temperatura de operação do motor, e à sua geometria complexa, assim como pela velocidade de prototipagem propiciada por esta tecnologia. O material de impressão usado foi ABS.\@

\section{Validação do projeto do motor}

O motor projetado e impresso em 3D foi montado em uma bancada de testes e instrumentado com os sensores da tabela ref. Os sensores, bem como uma válvula de gás comandada eletronicamente, foram ligados a um Arduino Mega para leitura e controle do conjunto de testes. 

%https://www.arduino.cc/reference/en/libraries/max6675/

\begin{table}[htbp]
    \centering\begin{tabular}{p{0.15\textwidth}p{0.5\textwidth}p{0.2\textwidth}} \toprule
        Sensor & Montagem & Biblioteca usada \\ \midrule
        Termopar & Inserido lateralmente na câmara de empuxo, selado com cola de ABS & MAX6675 \\
        Célula de carga & Apoio para o motor, sentido de medida paralelo ao empuxo & HX711 \\
        Transdutor de pressão & Acoplado à câmara de empuxo em furo lateral; ver CAD na seção RESULTADOS & Leitura analógica simples \\ \bottomrule
    \end{tabular}
    \caption{Descrição dos periféricos usados nos testes de validação do motor desenvolvido.}
    \label{tab:validation_peripherals}
\end{table}

Para o controle da bancada foi desenvolvida uma \textit{command line interface} simples para permitir testes interativos. Assim, as funções de tara, calibração, abertura e fechamento de válvula e leitura de sensores podiam ser comandadas a partir de uma interface textual interativa que facilitou a iteração do teste.

\section{Projeto do sistema de \textit{jet vanes}}

Devido à disponibilidade de materiais, foi utilizada uma lâmina de aço de 0,7mm de espessura como \textit{jet vane}. Com este sistema, desejou-se a capacidade de posicionar a lâmina defletora com resolução de \(1^\circ \), com alcance de deflexão de \(\pm 20^\circ \). Também foi necessário projetar um suporte para o motor que permitisse a conexão da mangueira de alimentação de gás. Por fim, este sistema deveria ter um encaixe circular de \(12mm\) para um eixo de acoplamento com a balança de três componentes.

\section{Caracterização do sistema em balança de três componentes}

\chapter{\textbf{RESULTADOS E DISCUSSÃO}}

\section{Projeto do motor}

Da metodologia e dos requisitos expostos na seção~\ref{sec:motor_project}, foi gerada a geometria do motor para impressão 3D. A figura~\ref{fig:internal_profile} mostra a seção longitudinal projetada para o motor. Destaca-se o grande volume da câmara em comparação com a tubeira, devido à necessidade de haver estagnação de um fluxo intenso (ver seção~\ref{sec:result_validation}).

\begin{figure}[htbp]
    \centering
    \includegraphics[width=\textwidth]{img/internal_profile.png}
    \caption{Perfil interno do motor projetado.}\label{fig:internal_profile}
\end{figure}

Os parâmetros propulsivos calculados para o motor foram:
\begin{itemize}
    \item \( \varepsilon = 1,35 \)
    \item \(C^* = 427,2\;\mathrm{m}\,\mathrm{s}^{-1}\)
    \item \(C_{F} = 1,10\)
\end{itemize}

Observa-se que o pequeno valor de razão de expansão é refletido na pequena tubeira exibida na figura~\ref{fig:internal_profile}. Foi obtido também um impulso específico de \(I_{sp} = 47,9s\) e um fluxo mássico de \(\dot{m} = 4,257\;\mathrm{g}\,\mathrm{s}^{-1}\). Devido à baixa energia do propelente, a velocidade característica do motor é bastante baixa, de modo que é necessário um fluxo mássico bastante elevado para a produção do empuxo requisitado. Isso é refletido no valor de impulso específico baixo. Ressalta-se também que o motor foi ajustado para operar à pressão ambiente, fator que limitou a razão de expansão e, portanto, o coeficiente de empuxo. 

O perfil da figura~\ref{fig:internal_profile} foi revolucionado para se obter uma geometria tridimensional, com superfícies planas externas adicionadas para facilitar a manufatura e a conexão com mangueiras de gás. Na figura~\ref{fig:3d_geom}, à esquerda, observa-se a geometria STL gerada em código para o motor. Foi adicionado um furo lateral para conexão com uma mangueira de alimentação, bem como quatro superfícies planas laterais para facilitar o manuseio e o apoio em morsas e outros equipamentos. À direita, observa-se o motor real construído \textcolor{red}{COLOCAR}. 

\begin{figure}[htbp]
    \centering
    \begin{subfigure}{0.49\textwidth}
        \includegraphics[width=\textwidth]{img/motor_stl.png}
        \caption{Geometria STL gerada para o motor.}
    \end{subfigure}
    \begin{subfigure}{0.49\textwidth}
        \includegraphics[width=\textwidth]{img/motor_stl.png}
        \caption{Motor impresso em 3D.}
    \end{subfigure}
    \caption{Geometria tridimensional do motor projetado.}
    \label{fig:3d_geom}
\end{figure}

\section{Validação do projeto do motor}\label{sec:result_validation}

A bancada de testes descrita na seção~\ref{sec:method_validation} gerou dados de empuxo para o motor e pressão estática e temperatura de câmara. A temperatura manteve-se bastante constante, próxima do valor especificado no requisito PRP-3, de modo que ela será tratada como constante e idêntica a este valor. O gráfico~\ref{fig:thrust_versus_p1} exibe as medidas de empuxo feitas em função da pressão de câmara medida.

\begin{figure}[htbp]
    \centering
    \includegraphics[width=\textwidth]{img/thrust_vs_chamber_pressure.png}
    \caption{Empuxo em função de pressão de câmara estática.}
    \label{fig:thrust_versus_p1}
\end{figure}

A variação de pressão de câmara observada neste ensaio deve-se ao grande fluxo mássico exigido pelo motor, de modo que houve esvaziamento significativo do compressor de ar utilizado durante o teste. Também destaca-se que a pressão medida na câmara, entre \(4,2bar\) e \(4,6bar\), foi inferior à pressão regulada em válvula, de exatamente \(5bar\). No entanto, o fato do motor real atingir os \(2N\) de empuxo exigidos com pressão próxima de \(4,5bar\) permitiu que a pesquisa continuasse levando-se em consideração as diferenças entre o motor real e o motor projetado.

Parâmetros propulsivos reais podem ser obtidos dos dados apresentados na figura~\ref{fig:thrust_versus_p1} através das equações~\ref{eq:C_F} e~\ref{eq:cstar}. Seus valores médios, com incerteza \(1\sigma \) são apresentados a seguir. A \(2N\) de empuxo, obteve-se impulso específico de \(46,6s\).
\begin{align}
    C_F &= 1,228 \pm 0,005 \\
    C^* &= (368,8 \pm 2,4)\;\mathrm{m}\,\mathrm{s}^{-1}
\end{align}

\section{Projeto do sistema de \textit{jet vanes}}

\section{Caracterização do sistema em balança de três componentes}

\chapter{\textbf{CONCLUSÃO}}
A proposta deste trabalho, de desenvolver um sistema de vetorização de empuxo e caracterizar seu desempenho, foi executada. Desenvolveu-se um motor foguete de gás frio de \(2\,\mathrm{N} \) que foi validado experimentalmente e então usado em um sistema de vetorização de empuxo com lâmina defletora (\textit{jet vane}). Por fim, este sistema foi caracterizado em relação à perda de empuxo causada pela adição da lâmina defletora e suas derivadas de controle foram obtidas. Apesar do objetivo ter sido atingido, foram encontradas várias dificuldades e fontes de erros experimentais que devem ser abordadas em trabalhos futuros.
% REFERENCIAS BIBLIOGRAFICAS
\renewcommand\bibname{\itareferencesnamebabel} %renomear título do capítulo referências
\bibliography{Referencias/referencias}

% Apendices
\appendix
\chapter{Histórico de desenvolvimento do motor foguete}
Neste apêndice apresentar-se-ão versões intermediárias do motor foguete desenvolvido, de modo a esclarecer algumas decisões tomadas ao longo do trabalho.

\begin{minipage}{.49\textwidth}
    \includegraphics[width=\textwidth]{img/app_dev_history/motor1.jpeg}
\end{minipage}
\begin{minipage}{.49\textwidth}
    Primeiro motor desenvolvido. Apenas uma câmara cilíndrica com uma tubeira em uma base e um furo roscado para a mangueira de gás na outra. Projetado para \(F = 5\;\mathrm{N}\), \(p_c = 5\;\mathrm{bar}\), \(\alpha_{\mathrm{div}} = 15\;\mathrm{^\circ}\), \(\alpha_{\mathrm{conv}} = 45\;\mathrm{^\circ}\) e razão de contração 6. Provou-se de difícil manuseio e a tubeira foi impressa em poucas camadas, de modo que sua precisão dimensional foi subótima.
\end{minipage}

\begin{minipage}{.49\textwidth}
    \includegraphics[width=\textwidth]{img/app_dev_history/motor2.jpeg}
\end{minipage}
\begin{minipage}{.49\textwidth}
    A razão de contração foi aumentada para 50, alongou-se a tubeira usando \(\alpha_{\mathrm{div}} = 15\;\mathrm{^\circ}\), a câmara foi alongada e foram adicionadas superfícies planas à base para facilitar o rosqueamento do conector de gás (base inferior, não visível na imagem). A cor encardida deve-se ao uso de cola de ABS para selar a peça, que apresentou grandes vazamentos de ar em testes preliminares.
\end{minipage}

\begin{minipage}{.49\textwidth}
    \includegraphics[width=\textwidth]{img/app_dev_history/motor3.jpeg}
\end{minipage}
\begin{minipage}{.49\textwidth}
    Adicionada entrada para conexão com um transdutor de pressão, ver seção~\ref{sec:method_validation}. O transdutor de pressão foi posicionado perpendicularmente à entrada de ar, na mesma posição da versão anterior, para medir a pressão estática na câmara.
\end{minipage}

\begin{minipage}{.49\textwidth}
    \includegraphics[width=\textwidth]{img/app_dev_history/motor4.jpeg}
\end{minipage}
\begin{minipage}{.49\textwidth}
    Ambas as conexões de gás agora são posicionadas em paredes laterais, e a geometria dos planos laterais foi mudada para quadrada, ao invés de hexagonal. A entrada de ar na base do motor dificultava sua montagem na bancada de teste de empuxo. O diâmetro das conexões de gás foi aumentado para reduzir perdas viscosas na tubulação, que estimavam-se ser altas devido à alta vazão.
\end{minipage}

\begin{minipage}{.49\textwidth}
    \includegraphics[width=\textwidth]{img/app_dev_history/motor5.jpeg}
\end{minipage}
\begin{minipage}{.49\textwidth}
    O empuxo do motor foi reduzido para \(2\;\mathrm{N}\). Neste ponto do trabalho, já havia-se testado os motores anteriores e foi consumida metade de um cilindro de nitrogênio comprimido. Visando reduzir custos e permitir testes mais extensivos, o empuxo foi reduzido para permitir o uso do compressor de baixa capacidade mencionado na~\ref{sec:discussion}. O extrusor usado a partir desse motor foi de \(0,2\;\mathrm{mm}\), ao invés de \(0,4\;\mathrm{mm}\) como nos motores anteriores, o que garantir maior precisão para a tubeira mais estreita. 
\end{minipage}

\begin{minipage}{.49\textwidth}
    \includegraphics[width=\textwidth]{img/app_dev_history/motor6.jpeg}
\end{minipage}
\begin{minipage}{.49\textwidth}
    Versão final do motor. Alongou-se o conjunto, reduzindo o ângulo de convergente para \(\alpha_{\mathrm{conv}} = 30\;\mathrm{^\circ}\) para reduzir a distância da tubeira à lâmina defletora. Foi removida a entrada para conexão com um transdutor de pressão, já que a conexão deste com o motor na balança introduziria ainda mais rigidez indesejada ao sistema.
\end{minipage}

% Anexos
\annex

% Glossario
%\itaglossary
%\printglossary

% Folha de Registro do Documento
% Valores dos campos do formulario
\FRDitadata{25 de março de 2015}
\FRDitadocnro{DCTA/ITA/DM-018/2015} %(o número de registro você solicita a biblioteca)
\FRDitaorgaointerno{Instituto Tecnológico de Aeronáutica -- ITA}
%Exemplo no caso de pós-graduação: Instituto Tecnol{\'o}gico de Aeron{\'a}utica -- ITA
\FRDitapalavrasautor{Cupim; Cimento; Estruturas}
\FRDitapalavrasresult{Propulsão; Gás Frio; Vetorização de empuxo;}
%Exemplo no caso de graduação (TG):
%\FRDitapalavraapresentacao{Trabalho de Graduação, ITA, São José dos Campos, 2015. \NumPenultimaPagina\ páginas.}
%Exemplo no caso de pós-graduação (msc, dsc):
\FRDitapalavraapresentacao{ITA, São José dos Campos. Curso de Mestrado. Programa de Pós-Graduação em Engenharia Aeronáutica e Mecânica. Área de Sistemas Aeroespaciais e Mecatrônica. Orientador: Prof.~Dr. Adalberto Santos Dupont. Coorientadora: Prof$^\textnormal{a}$.~Dr$^\textnormal{a}$. Doralice Serra. Defesa em 05/03/2015. Publicada em 25/03/2015.}
\FRDitaresumo{Este trabalho apresenta o processo de desenvolvimento e caracterização de um sistema de vetorização de empuxo com motor a gás frio. O motor tem como requisito empuxo de \(2\;\mathrm{N}\) e \(5\;\mathrm{bar}\) de pressão de câmara. O método de vetorização escolhido para teste foi o de \textit{jet vane}. O motor construído apresentou divergências pequenas com os requisitos, tendo um impulso específico de \(46,6\;\mathrm{s}\). Este motor foi montado em um mecanismo de controle da lâmina defletora e esta montagem foi acoplada a uma balança de três componentes para caracterização das forças e momentos gerados. Como resultado final, obtiveram-se as derivadas de controle de força lateral e momento. Por fim, apresentaram-se os problemas metodológicos encontrados e \textit{trade-offs} de engenharia identificados para o sistema.}
%  Primeiro Parametro: Nacional ou Internacional -- N/I
%  Segundo parametro: Ostensivo, Reservado, Confidencial ou Secreto -- O/R/C/S
\FRDitaOpcoes{N}{O}
% Cria o formulario
\itaFRD

\end{document}
% Fim do Documento. O massacre acabou!!! :-)
