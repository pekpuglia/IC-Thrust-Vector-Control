%% abtex2-modelo-artigo.tex, v-1.9.7 laurocesar
%% Copyright 2012-2018 by abnTeX2 group at http://www.abntex.net.br/
%%
%% This work may be distributed and/or modified under the
%% conditions of the LaTeX Project Public License, either version 1.3
%% of this license or (at your option) any later version.
%% The latest version of this license is in
%%   http://www.latex-project.org/lppl.txt
%% and version 1.3 or later is part of all distributions of LaTeX
%% version 2005/12/01 or later.
%%
%% This work has the LPPL maintenance status `maintained'.
%%
%% The Current Maintainer of this work is the abnTeX2 team, led
%% by Lauro César Araujo. Further information are available on
%% http://www.abntex.net.br/
%%
%% This work consists of the files abntex2-modelo-artigo.tex and
%% abntex2-modelo-references.bib
%%

% -----
% Versão adaptada para ENCITA
% Autoria Filipe Verri <verri@ita.br>
% Data 22/12/2021
% -----

% ------------------------------------------------------------------------
% ------------------------------------------------------------------------
% abnTeX2: Modelo de Artigo Acadêmico em conformidade com
% ABNT NBR 6022:2018: Informação e documentação - Artigo em publicação
% periódica científica - Apresentação
% ------------------------------------------------------------------------
% ------------------------------------------------------------------------

\documentclass[
	% -- opções da classe memoir --
	article,			% indica que é um artigo acadêmico
	10pt,				% tamanho da fonte
	oneside,			% para impressão apenas no recto. Oposto a twoside
	a4paper,			% tamanho do papel.
  twocolumn,			% para coluna dupla
	english,			% idioma adicional para hifenização
	brazil,				% o último idioma é o principal do documento
	sumario=tradicional,
	]{abntex2}


% ---
% PACOTES
% ---

% ---
% Pacotes fundamentais
% ---
\usepackage{lmodern}			% Usa a fonte Latin Modern
\usepackage[T1]{fontenc}		% Selecao de codigos de fonte.
\usepackage[utf8]{inputenc}		% Codificacao do documento (conversão automática dos acentos)
\usepackage{indentfirst}		% Indenta o primeiro parágrafo de cada seção.
\usepackage{nomencl} 			% Lista de simbolos
\usepackage{color}				% Controle das cores
\usepackage{graphicx}			% Inclusão de gráficos
\usepackage{microtype} 			% para melhorias de justificação
\usepackage{booktabs} % Tabelas
% ---

% ---
% Pacotes de citações
% ---
\usepackage[alf]{abntex2cite}	% Citações padrão ABNT
% ---

% --- Informações de dados para CAPA e FOLHA DE ROSTO ---
\titulo{Instruções para a formatação de trabalhos submetidos ao ENCITA}
\ifthenelse{\equal{\ABNTEXisarticle}{true}}{%
\renewcommand{\maketitlehookb}{}
}{}

\autor{
  Nome do primeiro autor\thanks{Nome da instituição, tipo de vínculo (bolsista PIBIC-CNPq,
  bolsista ITI-CNPq, pesquisador voluntário, orientador, professor, etc.),
  \href{mailto:fulano@ita.br}{fulano@ita.br}.},
  Nome do segundo autor\thanks{Informações do segundo autor}}

\local{Instituto Tecnológico de Aeronáutica, São José dos Campos, SP, Brasil}
\data{XXVII Encontro de Iniciação Científica do ITA -- XXVII ENCITA/2022}
% ---

% ---
% Configurações de aparência do PDF final

% alterando o aspecto da cor azul
\definecolor{blue}{RGB}{41,5,195}

% informações do PDF
\makeatletter
\hypersetup{
     	%pagebackref=true,
		pdftitle={\@title},
		pdfauthor={\@author},
		colorlinks=true,       		% false: boxed links; true: colored links
    	linkcolor=blue,          	% color of internal links
    	citecolor=blue,        		% color of links to bibliography
    	filecolor=magenta,      		% color of file links
		urlcolor=blue,
		bookmarksdepth=4
}
\makeatother
% ---

% ---
% compila o indice
% ---
\makeindex
% ---

% ---
% Altera as margens padrões
% ---
\setlrmarginsandblock{2cm}{2cm}{*}
\setulmarginsandblock{1.5cm}{2.5cm}{*}
\checkandfixthelayout
% ---

% ---
% Espaçamentos entre linhas e parágrafos
% ---

% O tamanho do parágrafo é dado por:
\setlength{\parindent}{0.7cm}

% Controle do espaçamento entre um parágrafo e outro:
\setlength{\parskip}{0.1cm}

% Espaçamento simples
\SingleSpacing

% Espaçamento após título das Referências
\setlength\afterchapskip{\lineskip}

% ----
% Início do documento
% ----
\begin{document}

% Seleciona o idioma do documento (conforme pacotes do babel)
\selectlanguage{brazil}

% Retira espaço extra obsoleto entre as frases.
\frenchspacing

% ----------------------------------------------------------
% ELEMENTOS PRÉ-TEXTUAIS
% ----------------------------------------------------------

% página de titulo principal (obrigatório)
\maketitle

% resumo em português
\begin{resumoumacoluna}
 Conforme a ABNT NBR 6022:2018, o resumo no idioma do documento é elemento obrigatório.
 Constituído de uma sequência de frases concisas e objetivas e não de uma
 simples enumeração de tópicos, não ultrapassando 250 palavras, seguido, logo
 abaixo, das palavras representativas do conteúdo do trabalho, isto é,
 palavras-chave e/ou descritores, conforme a NBR 6028. (\ldots) As
 palavras-chave devem figurar logo abaixo do resumo, antecedidas da expressão
 Palavras-chave:, separadas entre si por ponto e finalizadas também por ponto.
 O propósito deste ``template'' é servir como modelo de um trabalho a ser publicado nos
 anais do ENCITA. O resumo deve descrever os objetivos, a metodologia e as principais
 conclusões em não mais de 250 palavras. Ele não deve conter fórmulas nem deduções
 matemáticas. Figuras também não devem fazer parte do resumo.
 \vspace{\onelineskip}

 \noindent
 \textbf{Palavras-chave}: latex. abntex. editoração de texto.
\end{resumoumacoluna}
% ---

% ----------------------------------------------------------
% ELEMENTOS TEXTUAIS
% ----------------------------------------------------------
\textual

% ----------------------------------------------------------
% Introdução
% ----------------------------------------------------------
\section{Introdução}

Para inscrever-se no ENCITA o aluno deverá entregar um artigo para a coordenação
seguindo as instruções em \url{https://paic.ita.br/} em formato PDF.

Os Anais do ENCITA serão publicados no site da biblioteca do ITA. Para ser publicado, o
trabalho deve obedecer esta instrução, com respeito ao formato do texto e qualidade das
figuras e tabelas.

Os manuscritos devem ser submetidos em Português. O trabalho é limitado a 2 (duas)
páginas, incluindo tabelas, figuras, referências, etc.

Os nomes dos autores e afiliações devem aparecer já na 1ª versão submetida para revisão.
Caso o trabalho seja aceito sem modificações, os autores não precisarão enviar nova
versão.

Na introdução deve constar a justificativa do trabalho, contextualização e breve revisão
da literatura. Os objetivos do trabalho também devem estar no final da introdução.

\section{Seções do artigo}

O corpo de texto segue a mesma formatação da introdução. Sugere-se que as seções sejam: 1.
Introdução; 2. Material e Métodos; 3. Resultados e Discussão; 4. Conclusões e
Recomendações; 5. Agradecimentos; por fim, Referências.

\subsection{Corpo do texto}

As equações matemáticas devem ser citadas como Eq. \ref{eq:mass-energy} no meio da
frase, ou por Equação \ref{eq:mass-energy} no início de uma frase.

Os símbolos usados nas equações devem ser definidos imediatamente antes ou depois de sua
primeira ocorrência no texto do trabalho: a equivalência massa-energia é dada por
\begin{equation}
  \label{eq:mass-energy}
  E = m c^2\mbox{,}
\end{equation}
onde $E$ é energia, $m$ a massa e $c$ a velocidade da luz no vácuo.

O tamanho da fonte usado nas equações deve ser compatível com o utilizado no texto. Todos
os símbolos devem ter suas unidades expressas no S.I. (Sistema Internacional).

As figuras devem ser centralizadas e referenciadas como Fig. \ref{fig:example} no meio da
frase ou por Figura \ref{fig:example}, caso apareçam no início. As anotações e numerações
devem ter tamanhos compatíveis com o da fonte usada no texto, e todas as unidades devem
ser expressas no S.I. (Sistema Internacional). Cada figura deve ser colocada na posição
mais próxima possível de sua primeira citação no texto. As legendas das figuras devem ser
alinhadas à esquerda.

\begin{figure}[h]
  \begin{center}
    \includegraphics[width=0.9\columnwidth]{example-image-a}
  \end{center}
  \caption{Exemplo de uma figura.}
  \label{fig:example}
\end{figure}

Figuras coloridas e fotografias de alta qualidade podem ser incluídas no trabalho. É
recomendável que qualquer figura inserida no trabalho tenha ao menos 90 DPI.

\begin{table*}[t]
  \caption{Resultados experimentais para as propriedades de flexão dos materiais MAT1 and
  MAT2. Valores médios de obtidos em 20 ensaios.}
  \label{tab:large}
  \centering
  \begin{tabular}{ccc}
    \toprule
    Propriedades do compósito & CFRC-TWILL & CFRC-4HS \\
    \midrule
    Resistência à Flexão (MPa) & $209\pm10$ & $180\pm15$ \\
    Módulo de Flexão (GPa) & $57\pm3$ & $18\pm1$ \\
    \bottomrule
  \end{tabular}
\end{table*}

As tabelas devem ser centralizadas e referidas por Tab. \ref{tab:example} no meio da
frase, ou por Tabela \ref{tab:example} no início de uma sentença. Os títulos das tabelas
devem ser localizados imediatamente acima da tabela. Anotações e valores numéricos nela
incluídos devem ter tamanhos compatíveis com o da fonte usada no texto do trabalho, e
todas as unidades devem ser expressas no S.I. (Sistema Internacional). As unidades são
incluídas apenas na primeira linha/coluna, conforme for apropriado. As tabelas devem ser
colocadas tão perto, o quanto possível, de sua primeira citação no texto. O estilo de
borda da tabela é livre. Exemplos são apresentados na Tab. \ref{tab:example} e na Tab.
\ref{tab:large}.

\begin{table}[h]
  \caption{Propriedades após o processamento.}
  \label{tab:example}
  \centering
  \begin{tabular}{ccc}
    \toprule
    Experimento & Prop. 1 (\%) & Prop. 2 (m)\\
    \midrule
    Ensaio 1 & $40.0$ & $22.7$ \\
    Ensaio 2 & $48.4$ & $13.9$ \\
    \bottomrule
  \end{tabular}
\end{table}

A citação das referências no corpo do texto pode ser feita nos formatos:
\citeonline{ibge1993,araujo2012} mostram que \dots, ou: Observa-se que \dots
\cite{van86,guizzardi2005,masolo2010}.

Referências aceitáveis incluem: artigos de periódicos, dissertações, teses, artigos
publicados em anais de congressos, ``preprints'', livros e artigos submetidos e aceitos em
revistas (identificar a fonte). Não citar páginas web, material didáticos, artigos em
blogs, vídeos ou similares.

\section{Agradecimentos}

Se bolsista, não esquecer de agradecimento à agência de fomento.

% ----------------------------------------------------------
% Referências bibliográficas
% ----------------------------------------------------------
\bibliography{artigo-encita}

\end{document}
